\documentclass[a4paper,14pt]{article}
\usepackage{extsizes} % для размера 14
\usepackage{cmap} % для кодировки шрифтов в pdf
\usepackage[T2A]{fontenc}
\usepackage[utf8]{inputenc}
\usepackage[english, russian]{babel} % переносы
\usepackage{graphicx} % для вставки картинок
\usepackage{fontspec} % шрифт Times
\setmainfont{Times New Roman}
\usepackage{indentfirst} % отступ первого абзаца
\setlength{\parindent}{1.3cm} % отступ первой строки
\usepackage[onehalfspacing]{setspace}% полуторный интервал
\frenchspacing % пробел после двоеточия
\usepackage{geometry} % поля
\usepackage{float} % перенос текста после рисунков
\geometry{left=3cm}
\geometry{right=1.5cm}
\geometry{top=2cm}
\geometry{bottom=2cm}
\renewcommand{\labelitemi}{-} % тире в ненумерованных списках

\addto\captionsrussian{ % заголовок содержания
  \renewcommand{\contentsname}%
    {\centerline{\normalsize\textmd{СОДЕРЖАНИЕ}}}%
}

\graphicspath{{images/}} % путь к изображениям
\renewcommand{\thefigure}{\thesection.\arabic{figure}} % шаблон номера рисунков
\renewcommand{\thetable}{\thesection.\arabic{table}} % шаблон номера таблиц

\usepackage{longtable}
\usepackage[tableposition=top]{caption} % подпись таблицы вверху
\captionsetup[table]{skip=0pt,singlelinecheck=off} % выравнивание слева, многострочность заголовка таблицы
\DeclareCaptionLabelFormat{gostfigure}{Рисунок #2} %подпись рисунка
\DeclareCaptionLabelFormat{gosttable}{Таблица #2} %подпись таблицы
\DeclareCaptionLabelSeparator{gost}{~--~} %разделитель в рисунках и таблицах
\captionsetup{labelsep=gost}
\captionsetup[figure]{labelformat=gostfigure}
\captionsetup[table]{labelformat=gosttable}

\usepackage{titlesec}

\titleformat{\section} %настройка заголовка раздела
{\normalsize\bfseries}
{\thesection}
{1em}{}

\titleformat{\subsection}  %настройка заголовка подраздела
{\normalsize\bfseries}
{\thesubsection}
{1em}{}

\titleformat{\subsubsection}
{\normalsize\bfseries}
{\thesubsubsection}
{1em}{}

\titlespacing*{\section}{\parindent}{*2}{*2} % отступы разделов
\titlespacing*{\subsection}{\parindent}{*2}{*2}
\titlespacing*{\subsubsection}{\parindent}{*2}{*2}

\usepackage{array} % новый вид колонки с выравниванием по центру

\newcolumntype{C}[1]{>{\centering\let\newline\\\arraybackslash\hspace{0pt}}m{#1}}

\usepackage{enumitem}
\setlist{nolistsep} % отключить отступы вокруг списков

\usepackage{tocloft}
\renewcommand{\cftsecfont}{\normalsize} % не жирный шрифт раздела в оглавлении
\renewcommand{\cftsecpagefont}{\textmd} % не жирный номер страницы для раздела
\setcounter{tocdepth}{4} % число уровней оглавления
\setcounter{secnumdepth}{4} % число уровней подразделов

\addto\captionsrussian{\def\refname{СПИСОК ИСПОЛЬЗОВАННЫХ ИСТОЧНИКОВ}} %заголовок списка литературы
\makeatletter
\renewcommand\@biblabel[1]{#1.} %оформление элемента списка литературы

\renewenvironment{thebibliography}[1] %переопределение списка литературы, чтобы заголовок был не секцией
     {\begin{center}\refname\end{center}%
      \@mkboth{\MakeUppercase\refname}{\MakeUppercase\refname}%
      \list{\@biblabel{\@arabic\c@enumiv}}%
           {\settowidth\labelwidth{\@biblabel{#1}}%
            \leftmargin\labelwidth
            \advance\leftmargin\labelsep
            \@openbib@code
            \usecounter{enumiv}%
            \let\p@enumiv\@empty
            \renewcommand\theenumiv{\@arabic\c@enumiv}}%
      \sloppy
      \clubpenalty4000
      \@clubpenalty \clubpenalty
      \widowpenalty4000%
      \sfcode`\.\@m}


\makeatother
